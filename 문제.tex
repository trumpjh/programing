대양고등학교 프로그래밍 문제 모음
=====================================

[1일차 - 프로그래밍 언어 기초]
점수: /20

1. 프로그래밍 언어의 주요 목적에 대한 설명으로 가장 적절한 것은? [2점]
① 컴퓨터의 하드웨어를 직접 제어하는 것
② 다양한 운영체제에서 호환되는 프로그램을 개발하는 것
③ 기계와 의사소통하여 컴퓨터가 일을 할 수 있도록 명령을 내리는 것
④ 프로그램을 빠르게 실행할 수 있도록 최적화하는 것
⑤ 사용자가 이해하기 쉬운 형태로 데이터를 시각화하는 것
정답: ③

2. 파이썬 프로그래밍 언어의 특징으로 옳지 않은 것은? [2점]
① 간결하고 쉬운 문법을 가지고 있다.
② 수많은 표준 라이브러리를 제공한다.
③ 높은 생산성과 확장성을 가지고 있다.
④ 멀티스레드(multi-thread) 기능을 강력하게 지원한다.
⑤ 이식성이 뛰어나 다양한 운영체제에서 동작할 수 있다.
정답: ④

3. 프로그램을 작성하는 방식을 의미하며, 개발자가 프로그래밍 언어를 구성하는 방식을 이해하는 데 도움을 주는 개념은 무엇인가? [2점]
① 프로그래밍 언어
② 프로그래밍 패러다임
③ 저급 언어
④ 고급 언어
⑤ 프로그램
정답: ②

4. 컴퓨터가 이해하기 쉽고 0과 1로 명령어와 자료를 표현하는 언어를 무엇이라고 하는가? [2점]
① 고급 언어
② 객체 지향 언어
③ 함수형 언어
④ 저급 언어
⑤ 자연어
정답: ④

5. 다음 중 저급 언어에 속하는 프로그래밍 언어는? [2점]
① C 언어
② C++ 언어
③ 자바 언어
④ 기계어
⑤ 파이썬 언어
정답: ④

6. 다음 중 인공 지능(AI) 및 머신 러닝 개발에 가장 적합한 프로그래밍 언어는? [2점]
① C 언어
② C++ 언어
③ 자바 언어
④ 기계어
⑤ 파이썬 언어
정답: ⑤

7. 고급 언어에 대한 설명으로 옳은 것은? [2점]
① 하드웨어를 직접 제어할 수 있어 실행 속도가 매우 빠르다.
② 기계 친화적이므로 배우기 어렵다.
③ 컴퓨터가 직접 이해할 수 있어 별도의 번역 프로그램이 필요 없다.
④ 사람이 사용하는 기호 체계를 이용하는 사람 중심의 언어이다.
⑤ 기계어와 어셈블리어가 대표적인 고급 언어에 속한다.
정답: ④

8. 프로그래밍 패러다임 중 "프로그램을 수많은 객체(object)라는 기본 단위로 나누고 이들을 상호 작용으로 설계한다"는 특징을 가진 것은? [2점]
① 절차 지향 프로그래밍
② 구조적 프로그래밍
③ 객체 지향 프로그래밍
④ 함수형 프로그래밍
⑤ 명령형 프로그래밍
정답: ③

9. 다음 중 원시 프로그램 전체를 한꺼번에 번역하여 목적 프로그램을 생성하며, 한 번 목적 프로그램이 생성되면 재실행 시 번역 과정을 거치지 않아 실행 속도가 빠른 프로그래밍 구현 방법은? [2점]
① 인터프리터
② 하이브리드
③ 컴파일러
④ 디버깅
⑤ 로더
정답: ③

10. 자바스크립트 언어의 특징으로 옳지 않은 것은? [2점]
① 웹브라우저에 인터프리터가 내장되어 있는 인터프리터 언어이다.
② 웹 페이지에 효과를 주거나 기능을 향상하는 목적으로 사용된다.
③ 멀티 패러다임 언어로 함수형, 명령형, 객체 지향형 언어를 지원한다.
④ 비동기 프로그래밍에 불리하며 코드를 이해하기 쉽다.
⑤ 버전에 따라 실행되지 않는 기능이 많고 CPU 작업이 많을 때 비효율적일 때가 있다.
정답: ④

=====================================

[2일차 - 개발 절차 및 모형]
점수: /20

1. 프로그램 개발 절차를 체계적으로 진행해야 하는 가장 큰 이유는? [2점]
① 개발자의 재량권을 최대화하기 위해
② 최신 기술을 무조건 적용하기 위해
③ 해결해야 하는 문제의 요구 사항을 효율적이면서도 명확하게 처리하기 위해
④ 개발 기간을 최대한 늘려 안정성을 확보하기 위해
⑤ 최소한의 인력으로 개발을 완료하기 위해
정답: ③

2. 전통적으로 많이 사용되어 왔으며, 전체적인 방향성과 추진 상황을 명확하게 파악할 수 있고 개발 일정과 예산을 예측하기 쉬운 프로그램 개발 절차 모형은? [2점]
① V 모형
② 애자일 모형 (Agile Model)
③ 나선형 모형 (Spiral Model)
④ 폭포수 모형
⑤ 반복적 모형 (Iterative Model)
정답: ④

3. 프로그램 개발 절차에서 문제나 사용자 요구 사항을 이해하고, 이를 해결하기 위한 해결 방법을 찾는 과정은 무엇인가? [2점]
① 설계
② 구현
③ 테스트
④ 문제 분석
⑤ 요구 분석
정답: ④

4. 문제 분석 단계에서 파악한 문제를 해결하기 위해 필요한 기능과 특성을 요구 사항으로 정리하는 과정은? [2점]
① 설계
② 구현
③ 테스트
④ 문제 분석
⑤ 요구 분석
정답: ⑤

5. 프로그램의 구조, 각 모듈 간의 연결, 데이터 구조 등을 정의하며, 충분한 시간을 투자하고 적절한 도구와 방법을 활용하여 수행하는 것이 중요한 개발 단계는? [2점]
① 설계
② 구현
③ 테스트
④ 문제 분석
⑤ 요구 분석
정답: ①

6. 설계 단계에서 고려해야 할 사항으로 옳지 않은 것은? [2점]
① 개발자가 직접 작성한 코드나 시스템의 오류를 발견하고 수정하는 과정
② 프로그램을 모듈화하여 코드의 재사용성과 유지 보수성을 높이는 방안 정의
③ 시스템에서 다룰 데이터의 구조와 처리 방법을 결정
④ 구현해야 할 기능의 알고리즘을 선택하고 최적화 방법을 결정
⑤ 프로그램의 기능, 성능, 보안, 확장성 등에 대한 계획 수립
정답: ①

7. 개별 모듈이나 함수 등의 단위 코드를 테스트하며, 일반적으로 개발자가 직접 작성하여 실행하는 테스트 유형은? [2점]
① 단위 테스트
② 시스템 테스트
③ 인수 테스트
④ 통합 테스트
⑤ 회귀 테스트
정답: ①

8. 사용자나 고객의 입장에서 실제 환경에서 사용될 때의 동작을 검증하는 테스트로, 대표적으로 알파 테스트와 베타 테스트가 있는 테스트 유형은? [2점]
① 단위 테스트
② 시스템 테스트
③ 인수 테스트
④ 통합 테스트
⑤ 회귀 테스트
정답: ③

9. 프로그램 시스템 전반에 대한 테스트로 사용자 시나리오를 기반으로 테스트를 수행하며, 일반적으로 개발자나 테스터가 진행하는 테스트 유형은? [2점]
① 단위 테스트
② 시스템 테스트
③ 인수 테스트
④ 통합 테스트
⑤ 회귀 테스트
정답: ②

10. 세계 최초의 컴퓨터 프로그래머로, 1843년에 기계가 특정한 식을 계산할 수 있도록 하는 '명령문' 예시를 개발하고 오늘날 프로그래밍 언어의 필수적인 루프문, 조건문, 서브루틴 등 프로그램 제어문 개념을 최초로 고안한 인물은? [2점]
① 찰스 배비지
② 리누스 토발스
③ 귀도 반 로섬
④ 에이다 러브레이스
⑤ 앨런 튜링
정답: ④

=====================================

[3일차 - 알고리즘 기초]
점수: /20

1. 어떤 문제를 컴퓨터로 해결하기 위한 논리적인 절차나 방법을 의미하며, 주어진 문제를 처리하면 항상 올바른 방법으로 해결되어야 하는 것은 무엇인가? [2점]
① 프로그램
② 프로그래밍 언어
③ 알고리즘
④ 프로그래밍 패러다임
⑤ 데이터 구조
정답: ③

2. 알고리즘이 만족해야 하는 조건으로 옳지 않은 것은? [2점]
① 입력: 0개 이상의 입력이 외부에서 제공되어야 한다.
② 출력: 입력에 대해 최소한 1개 이상의 결과가 존재해야 한다.
③ 명확성: 각 단계의 의미가 명확해야 한다.
④ 유한성: 각 단계는 유한해야 한다(반드시 종료되어야 한다).
⑤ 효율성: 모든 과정이 무한한 시간 내에 실행 가능해야 한다.
정답: ⑤

3. 동일한 문제를 해결하는 여러 알고리즘 중에서 어떤 것이 가장 적절한 방법인지를 선택하기 위한 평가 기준으로, 알고리즘에서 처리하는 연산의 횟수(실행 시간)를 의미하는 것은? [2점]
① 정확성
② 유한성
③ 명확성
④ 공간 복잡도
⑤ 시간 복잡도
정답: ⑤

4. 알고리즘의 평가 방법 중 "입력 데이터양에 따른 알고리즘 처리 시간의 예측이 가능해진다"는 점에서 특히 중요하게 다루어지는 것은? [2점]
① 정확성
② 유한성
③ 명확성
④ 공간 복잡도
⑤ 시간 복잡도
정답: ⑤

5. 알고리즘을 표현하는 대표적인 방법 중, 국제표준기호(ISO)를 사용하여 순서를 나타내는 방법은? [2점]
① 의사 코드
② 자연어
③ 프로그래밍 언어
④ 순서도 (flow-chart)
⑤ 텍스트 기반 코드
정답: ④

6. 알고리즘을 표현하는 방법 중, 흐름을 텍스트로 표현하며 특정 프로그래밍 언어에 구애받지 않는 것은? [2점]
① 순서도
② 자연어
③ 프로그래밍 언어
④ 의사 코드
⑤ UML 다이어그램
정답: ④

7. 순서도에서 사용하는 기호 중 "조건에 따라 참(True), 거짓(False)으로 처리"하는 기능을 나타내는 기호의 명칭은? [2점]
① 단말
② 판단
③ 처리
④ 입출력
⑤ 준비
정답: ②

8. 순서도에서 사용하는 기호 중 "순서도의 시작과 끝을 알림" 기능을 나타내는 기호의 명칭은? [2점]
① 단말
② 판단
③ 처리
④ 입출력
⑤ 준비
정답: ①

9. 다음 순서도 구조 중, 조건에 따라 명령을 반복 수행할 수 있는 것은? [2점]
① 순차 구조
② 선택 구조
③ 반복 구조
④ 병렬 구조
⑤ 분산 구조
정답: ③

10. 사각형의 넓이를 계산하는 알고리즘을 의사 코드로 표현할 때, 입력으로 필요한 것은? [2점]
① 넓이(S)
② 가로(w), 세로(h)
③ 시작과 끝
④ 계산식 (w * h)
⑤ 출력 결과
정답: ②

=====================================

[4일차 - 개발 환경 및 도구]
점수: /20

1. 코드를 작성하고 편집하며 오류를 확인하는 등의 작업을 하나의 편집기 안에서 할 수 있는 통합 애플리케이션으로, 효율적으로 소프트웨어를 개발할 수 있도록 도와주는 환경은? [2점]
① 운영체제 (OS)
② 통합 개발 환경 (IDE)
③ 버전 관리 시스템 (VCS)
④ 컴파일러
⑤ 인터프리터
정답: ②

2. 디지털 사회에서 주어진 문제를 디지털 기술을 활용하여 해결하고, 다른 사람과 소통 및 협업하며 정보를 탐색하고 분석하여 결과물을 생산하는 능력을 무엇이라고 하는가? [2점]
① 디지털 문해력
② 디지털 전환
③ 디지털 역량
④ 정보 통신 기술 (ICT)
⑤ 인공 지능 (AI)
정답: ③

3. 언론인 폴 길스터가 1997년에 처음 소개한 개념으로, 인터넷에서 찾아낸 다양한 정보를 자신의 목적에 맞는 새로운 정보로 조합하여 올바로 사용하는 능력을 의미하는 것은? [2점]
① 디지털 역량
② 디지털 기술
③ 디지털 문해력 (디지털 리터러시)
④ 인공 지능 (AI)
⑤ 사물 인터넷 (IoT)
정답: ③

4. 2005년 리눅스 제작자인 리누스 토발스가 공개 프로그램 리눅스 커널 개발의 효율성을 높이기 위해 개발한 분산형 버전 관리 시스템으로, 코드 수정 권한, 버전 추적 등 소프트웨어 개발에 필요한 관리 기능을 제공하는 것은? [2점]
① Git
② Google Workspace
③ GitHub
④ Microsoft Teams
⑤ Jupyter Notebook
정답: ①

5. 파이썬 프로그래밍 언어의 개발자로 올바른 것은? [2점]
① 리누스 토발스
② 빌 게이츠
③ 스티브 잡스
④ 귀도 반 로섬
⑤ 제임스 고슬링
정답: ④

6. 파이썬의 통합 개발 및 학습 환경(IDLE)에서 긴 코드를 작성할 때 사용하기에 적합하지 않은 모드는? [2점]
① 에디터 모드
② 프로젝트 모드
③ 셸 모드
④ 디버그 모드
⑤ 비주얼 모드
정답: ③

7. 파이썬 프로그래밍에 특화된 통합 개발 환경으로, 스마트 코드 완성, 즉석 오류 검사 및 빠른 수정, 자동 코드 리팩터링 등 다양한 기능을 제공하는 것은? [2점]
① IDLE
② Jupyter Notebook
③ Colab
④ VS Code
⑤ PyCharm
정답: ⑤

8. 구글에서 제공하는 주피터 노트북 개발 환경으로, 별도의 구성 없이 웹 브라우저에서 파이썬을 작성하고 실행할 수 있으며, 자주 사용되는 라이브러리가 기본적으로 설치되어 있고, 가상 서버에 있는 GPU를 무료로 사용할 수 있는 환경은? [2점]
① IDLE
② Jupyter Notebook
③ Colab
④ VS Code
⑤ PyCharm
정답: ③

9. 파이썬을 설치할 때, 파이썬 인터프리터를 실행하기 위한 명령어를 어느 경로에서든 실행할 수 있도록 반드시 선택해야 하는 옵션은? [2점]
① Add python.exe to PATH
② Install launcher for all users
③ Customize installation
④ Disable path length limit
⑤ Install for all users
정답: ①

10. 코랩에서 다른 사람과 파일을 공유할 때, 상대방이 파일을 단순하게 조회만 할 수 있도록 허용하는 공유 모드는? [2점]
① 편집자 모드
② 뷰어 모드
③ 댓글 작성자 모드
④ 관리자 모드
⑤ 개발자 모드
정답: ②

=====================================

문제 통계:
- 총 4일차 과정
- 각 일차당 10문제 (총 40문제)
- 각 문제당 2점 (일차당 총 20점)
- 전체 총점: 80점

주제별 분류:
1일차: 프로그래밍 언어의 기본 개념, 특징, 패러다임, 구현 방식
2일차: 소프트웨어 개발 생명주기, 테스팅, 프로그래밍 역사
3일차: 알고리즘 정의, 조건, 복잡도, 표현 방법, 순서도
4일차: 개발 도구, IDE, 디지털 역량, 버전 관리, 파이썬 환경

=====================================

GitHub Pages 배포 가이드:

1. GitHub 저장소 생성
   - 저장소 이름: programming-quiz 또는 원하는 이름
   - Public 저장소로 설정

2. 파일 업로드
   - index.html (메인 페이지)
   - style.css (스타일시트)
   - questions.js (문제 데이터)
   - 문제.txt (문제 텍스트 파일 - 선택사항)

3. GitHub Pages 설정
   - 저장소 Settings > Pages
   - Source: Deploy from a branch
   - Branch: main / master 선택
   - Folder: / (root) 선택

4. 접속 URL
   - https://[username].github.io/[repository-name]

기능 특징:
- 반응형 디자인 (모바일, 태블릿, PC 지원)
- 일차별 학습 진행
- 실시간 점수 계산
- 오답 노트 기능
- 진행률 표시
- 애니메이션 효과
- 다크모드 지원
- 접근성 고려 설계

사용 방법:
1. 홈페이지 접속
2. 학습하고 싶은 일차 선택 (1-4일차)
3. 퀴즈 시작하기 버튼 클릭
4. 문제를 읽고 정답 선택
5. 다음 문제로 이동 또는 이전 문제로 돌아가기
6. 모든 문제 완료 후 결과 확인
7. 오답이 있을 경우 오답 확인 기능 이용
8. 다른 일차 학습을 위해 다시 시작

=====================================